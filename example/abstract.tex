% (This file is included by thesis.tex; you do not latex it by itself.)

\begin{abstract}

% The text of the abstract goes here.  If you need to use a \section
% command you will need to use \section*, \subsection*, etc. so that
% you don't get any numbering.  You probably won't be using any of
% these commands in the abstract anyway.

Analog and mixed signal IC design is notoriously difficult and slow due in large part to the layout. Modern integrated circuit fabrication with such small devices can have significant interconnect parasitics that can drastically affect the behavior of a circuit's design. The implication is that simulations of circuit's behavior are unreliable until after the interconnect parasitics are extracted from the layout and included in the simulation. 

The Berkeley Analog Generator (BAG) is a Python-based tool that interfaces with the Cadence Virtuoso software \cite{chang_bag2:_2018} that aims to solve the above problem. BAG allows the user to write parametrizable generator scripts that will automatically generate the entire layout and schematic, as well as run the layout-versus-schematic (LVS) and post-layout extraction (PEX) tools and export the results in a time that ranges from seconds to minutes based on circuit complexity. Designers who have decided on a certain topology can write a layout and schematic generator script in a high level programming language with class based hierarchy once, and then any changes in the circuit simply require changing the corresponding parameters file containing the circuit specifications. Additionally, BAG allows the automation of simulation and post-processing of simulation data as well as implementation of higher-level design scripts that encapsulate designer insights and methodology, as well as opens the doors for automated optimizer-driven circuit design.

This report shows examples of many common circuit blocks and their BAG implementation in an advanced process node, as well as an example of how BAG can be used to speed up the design process. Although the generator scripting offers the implementation in a higher-level language, certain implementation strategies and methodologies work better than others, and this report aims at presenting a systematic generator writing methodology and illustrates it on a set of typical analog-mixed signal blocks found in a high-speed link front-end. In three months, a library of generators ranging from small basic circuits to entire receiver chains were written; then in roughly two weeks, an LVS/PEX verified design for a 25Gbps optical communication link receiver in a 14nm FinFET process was created using BAG cells and test benches. Further possibilities and uses of BAG are also discussed.

\end{abstract}