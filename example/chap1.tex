\chapter{What Makes Analog Design Difficult?}

If you know anyone who fits into the category of analog, RF or mixed signal designers, you may already be familiar with the disheartened, defeated look they often display. Is it that IC design attracts people of this nature? Or is it that IC design slowly gnaws at the existence of engineers until they are but a shell of their former selves? For this report, we will assume the latter. We will also assume this is due to standard practices more than any deep-rooted truths about IC design or people in general.

\section{Modern Mixed Signal IC Flow and Concerns}
IC design is slow. The fabrication process for an IC is very complex, and for complicated chips the upfront cost can be immense \cite{elder_real_nodate}. Mistakes are costly both monetarily and time-wise, ranging from hundreds of thousands of dollars and months to millions of dollars for simple edits to a circuit. Furthermore, a company selling chips for use in other companies' products are liable for their chip's performance. Should their component be the reason for a product's failure, that company may then be responsible for reimbursing the cost of the product. For a \$1000 cell phone with a million units shipped per year, a 1\% chip failure rate can lead to costs in the tens of millions. For a company designing ICs then, it is crucial to get a high yield of functioning chips on the first try to minimize costs.

Since a chip cannot be tested until after it is manufactured, IC designers tend to simulate the worst possible scenarios. Semiconductors behave quite differently under varying temperatures and circuit conditions may not be stable. Supply voltages may vary, manufacturing can create component offset that negatively impacts the circuit, static discharge events can occur. Due to this, most companies will only tape out a chip when they are fully confident it will function properly under all use cases. 

Ignoring the design procedure and assuming an architecture and component parameters are already chosen, the testing procedure is generally as follows:
\begin{itemize}
\item[1.] Simulate the chosen design across all specs to ensure proper behavior
\item[2.] Simulate again, but at extreme temperatures and supply changes
\item[3.] Simulate again, but including random process variation
\item[4.] Draw the layout and extract the parasitics
\item[5.] Perform 1. through 3. on the extracted layout
\end{itemize}

% cite with \cite{name}

If the circuit fails at any of these steps, the design must be modified and the steps repeated. Only when all of tests are passed and verified through potentially hundreds of different tests will the chip be sent for fabrication. Depending on the complexity of the chip and the size of the team, this process can take anywhere from months to a year or more.
\subsection{Why Does it Take So Long?}
Layout.

\subsection{Why Does Layout Take So Long?}
Laying out a circuit is the term used to describe the process of how an abstract schematic is translated into a physical picture of where a fabrication facility should dope the semiconductor, add connections and place and route metals to generate an accurate depiction of the desired device parameters and sizes. This means that the designer must manually draw active regions, metal wire sizes and routing, vias, gate connections, etc. There are an infinite number of possibilities to generate such a layout, but each comes with some trade off. Perhaps making a wire very long and routing it around a structure is much easier, but introduces significant IR drop. Perhaps a wire needs to carry more current through it, so the width is increased, but this introduces extra parasitic capacitance. These trade offs are some of the main concerns of layout engineers. 

Layout bottlenecks the design flow due to the slow turnaround time. With the reduction in device sizes over the years, the design rules imposed by the fabricators have become more strict and the parasitics in conjunction with high frequencies have started to have a more significant effect on the circuit's post-layout performance \cite{allen_practice_2004} \cite{lourenco_layout-aware_2015}. Layout then, is a crucial part of the verification step. Since the process is slow and precise, this means that any changes require significant work to modify the layout and ensure nothing was incorrectly altered in the process.

The time frame of months to years is believable then if the turnaround for hand-drawn layout is roughly a couple days to a week for fairly insignificant design modifications like transistor resizing, or passive device resizing. Each and every time the designer receives the extracted layout and modifies the components to correct for changes, they must wait days before they can start the cycle again. By Amdahl's law then \footnote{Amdahl's law is generally referenced in computer program runtime, but the concept of speeding up fractions of a process applies here as well.} \cite{noauthor_amdahls_nodate}, speeding up the layout should allow one to close the design loop faster and reduce the pain of IC process.
